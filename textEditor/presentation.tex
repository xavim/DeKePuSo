\documentclass{beamer}

\usetheme{AnnArbor}

\title{Ser eficient amb l'editor de text}
\subtitle{(o per què uso l'editor vi)}

\author{Juanjo Costa}
\institute{jcosta@ac.upc.edu}
\date[29/11/23]{Barcelona, 29 Novembre 2023}

\begin{document}


% Title page frame
\begin{frame}
    \titlepage
\end{frame}

% Outline frame
\begin{frame}{Index}
    \tableofcontents
\end{frame}

% Show outline at each section : Current section
\AtBeginSection[ ]
{
\begin{frame}{Outline}
    \tableofcontents[currentsection]
\end{frame}
}

% Presentation structure
%\section{Per què dedicar temps a l'editor? Exemple d'us}
%\section{Vi: un editor 'diferent'}
%\section{Basics}
%    \subsection{Com em moc?}
%    \subsection{Com afegeixo text?}
%    \subsection{Com modifico el text?}
%    \subsection{Com busco text?}
%    \subsection{Com copio text?}
%\section{Advanced}
%    \subsection{Gestió de finestres}
%    \subsection{Gestió de buffers}
%    \subsection{Ajudes a la programant}
%    \subsection{No et repeteixis}
%\section{Conclusió}
%\section{Com puc saber-ne més?}


\section{Per què dedicar temps a l'editor? Exemple d'us}
\begin{frame}{Exemple d'ús}
\end{frame}
\section{Vi: un editor 'diferent'}
    \begin{frame}{Vi: un editor 'diferent'}
        \begin{itemize}
            \item Editor modal
                \begin{description}
                    \item [Inserció] Mode en el que \textbf{NOMÉS} podem escriure text
                    \item [Comandes] Mode en el que ens podem moure i executar
                        comandes (per defecte)
                \end{description}
                \begin{alertblock}{Diferència vs editor habitual}
                    Normalment els editors s'executen en mode
                    "Inserció"... però el que fem habitualment és ...
                    \textbf{VI}sualitzar  :) i esporàdicament fer algun canvi.
                \end{alertblock}
        \end{itemize}
    \end{frame}

\section{Basics}
    \subsection{Com em moc?}
    \begin{frame}{Com em moc?}
        \begin{itemize}
            \item Moviment
                \begin{itemize}
                    \item Caràcter \alert{h j k l}
                    \item Paraules inici final \alert{w b e}
                    \item Línia  inici final \alert{0 \$}
                    \item Altres \alert{ctrl-u ctrl-d }
                    \item Inici final fitxer \alert{gg G}
                    \item On soc? \alert{ctrl-g}
                \end{itemize}
            \item Sortir de vim
                \alert{:q! ZZ}
            \item Ajuda {:help}
            \begin{alertblock}{Useful options}
                \begin{itemize}
                    \item Ruler (\alert{:set ruler})
                \end{itemize}
            \end{alertblock}
            \onslide<1->
            \begin{block}{El moviment és el més important}
                Moure'ns eficientment pel fitxer! Només superat per la cerca :)
            \end{block}
        \end{itemize}
    \end{frame}

    \subsection{Com afegeixo text?}
    \begin{frame}{Com afegeixo text?}
        \begin{itemize}
            \item Entrar en mode insercio \alert{i I a A o O}
            \item Sortir del mode insercio \alert{Esc Ctrl-3}
        \end{itemize}
    \end{frame}
    \subsection{Com modifico el text?}
    \begin{frame}{Com modifico el text?}
        \begin{itemize}
            \item Borrar
                \alert{"composició de comandes"}
                \begin{itemize}
                    \item Caràcter  \alert{x}
                    \item Paraula   \alert{dw d3w}
                    \item Línia   \alert{dd}
                    \item Fins a final de linia \alert{d\$}
                \end{itemize}
            \item Replace caràcter \alert{r} o fins que em cansi \alert{R}
            \item Change \alert{cw c3w}
            \item Join lines \alert{J}
            \item M'he equivocat \alert{u} ... no, no, no m'havia equivocat \alert{ctrl-R}
        \end{itemize}
    \end{frame}

    \subsection{Com busco text?}
    \begin{frame}{Com busco text?}
        \begin{itemize}
            \item Cerca caràcter \alert{f caràcter} i next \alert{;}
                \begin{itemize}
                    \item Cercar caràcter (posició anterior) \alert{t caràcter}
                    \item Utilitzar majuscules canvia direcció cerca \alert{F T}
                \end{itemize}
            \item Cerca paraula \alert{/}
                \begin{itemize}
                    \item Next \alert{n} and previous \alert{N}
                \end{itemize}
            \item Cerca paraula actual \alert{*}
            \item Match parenthesis \alert{\%}
            \item Substitució \alert{:s/pattern/subst/} i repetir ultima substitució \alert{\&}
            \begin{alertblock}{Useful options}
                \begin{itemize}
                    \item Incremental search \alert{:set incsearch}
                    \item Highlight search \alert{:set hlsearch}
                    \item Show Match \alert{:set showmatch}
                \end{itemize}
            \end{alertblock}
        \end{itemize}
    \end{frame}

    \subsection{Com copio text?}
    \begin{frame}{Com copio text?}
        \begin{itemize}
            \item Copy/\emph{Yank} line \alert{yy}
            \item Paste \alert{p}
            \item Seleccio de text \alert{v} \alert{ctrl-v}
        \end{itemize}
            \begin{alertblock}{Composició de comandes}
                Recorda que pots compossar comandes \alert{10yy y2w}
            \end{alertblock}
    \end{frame}

\section{Advanced}
    \subsection{Gestió de finestres}
    \begin{frame}{Gestió de finestres}
        \begin{itemize}
            \item Split windows (horitzontal / vertical) \alert{:sp :vsp}
            \item Change windows \alert{Ctrl-w h j k l}
            \item Change window size
                \begin{itemize}
                    \item Increase/decrease \alert{Ctrl-w + --}
                    \item Maximize \alert{Ctrl-w \_ }
                    \item Equal size windows \alert{Ctrl-w = }
                \end{itemize}
        \end{itemize}
    \end{frame}

    \subsection{Gestió de buffers}
    \begin{frame}{Gestió de buffers}
        \begin{itemize}
            \item Llistar buffers \alert{:ls}
            \item Canviar buffer \alert{:[N]b}
        \end{itemize}
    \end{frame}

    \subsection{Ajudes a la programació}
    \begin{frame}{Ajudes a la programació}
        \begin{itemize}
            \item Vola pel codi: \alert{Ctags}
                \begin{block}{Ctags}
                    Ctags és una eina per indexar funcions, variables, ... d'un
                    codi en C
                \end{block}
                \begin{itemize}
                    \item Saltar al tag \alert{Ctrl-]}
                    \item Tornar Enrera \alert{Ctrl-t}
                    \item Navegar dins llista tags \alert{Ctrl-O (jump back) Ctrl-I (jump Forward)}
                \end{itemize}
            \item make \alert{:make :cn :cp :cl}
            \item Consulta el man de la comanda actual \alert{K}
        \end{itemize}
    \end{frame}

    \subsection{No et repeteixis}
    \begin{frame}{No et repeteixis}
        \begin{itemize}
            \item Macros
                \begin{itemize}
                    \item Començar gravació macro \alert{q 'registre'}
                    \item Finalitzar gravació macro \alert{Esc}
                    \item Executar macro \alert{@ 'registre'}
                    \item Executar última macro \alert{@ @}
                \end{itemize}
            \item Autocomplete \alert{Ctrl-N  Ctrl-P}
            \item Repeat last command \alert{.}
        \end{itemize}
    \end{frame}

    %- vimdiff

    %- indent > <

    %- gq
\section{Conclusió}
    \begin{frame}{Conclusió}
        \begin{itemize}
            \item Com ser eficient amb l'editor de text?
                \onslide<1->
                \begin{itemize}
                    \item <2-> Cal ser ràpid
                    \item <3-> Cal repetir les comandes bàsiques fins que siguin un reflex
                    \item <4-> Cal detectar coses repetitives i intentar reduir el temps
                    \item <5-> Iterar i gaudir
                \end{itemize}
            \item <6->Cal utilitzar 'vi' ?
            \item <7->No cal, però sí minimitzar la fricció amb el teu editor
                \onslide<8->
                \begin{itemize}
                    \item Vi està present a qualsevol entorn unix/linux.
                    \item Vi és ràpid.
                    \item Vi ocupa pocs recursos.
                \end{itemize}
                \onslide<9->
                \begin{block}{Això és tot?}
                    No hem ni començat... però amb això podeu ser més eficients.
                \end{block}
        \end{itemize}
    \end{frame}

\section{Com puc saber-ne més?}
    \begin{frame}{Com puc saber-ne més?}
        \begin{itemize}
            \item VIMTUTOR (per començar)
            \item Seven habits of effective text editing. Bram~Moolenaar. November~2000. \url{https://www.moolenaar.net/habits.html}
            \item Ajuda del propi VIM \alert{:help} (per buscar funcionalitats noves)
            \item VIM Homepage. \url{https://www.vim.org/}
            \item Quick reference guide. \url{https://vimhelp.org/quickref.txt.html}
        \end{itemize}

    \end{frame}

\begin{frame}{Moltes gràcies}
    Happy hacking! :)
\end{frame}

\begin{frame}
    \titlepage
\end{frame}
\end{document}
